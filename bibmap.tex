
% !Mode:: "TeX:UTF-8"
% 用于
\documentclass{article}

\input{bibmap-preamble} %宏包和一些格式设置
%\usepackage{bibmap}%[citestyle,bibstyle,mapstyle]

\lstnewenvironment{pycode}%
{\lstset{% general command to set parameter(s)
%name=#1,
%label=#2,
%caption=\lstname,
linewidth=\linewidth,
breaklines=true,
%showspaces=true,
extendedchars=false,
columns=fullflexible,%flexible,
aboveskip=2pt,
boxpos=t,
rulesep=0pt,
frame=tb,
framesep=0pt,
rulecolor=\color{gblabelcolor},
fontadjust=true,
language=Python,
backgroundcolor=\color{gbyellow!3},%\color{yellow}, %背景颜色
numbers=left,
numberstyle=\tiny\color{gblabelcolor},
basicstyle=\footnotesize\ttfamily, % print whole listing small
keywordstyle=\bfseries\color{gbemphcolor},%\underbar,
% underlined bold black keywords
identifierstyle=, % nothing happens
commentstyle=\color{green!40!gray}, % white comments
stringstyle=\ttfamily\color{purple!50}, % typewriter type for strings
showstringspaces=false}% no special string spaces
}
{}


\begin{document}

\section{介绍}

\subsection{bibmap是什么}

bibmap是一个用于处理参考文献的latex宏包,
包含一个sty文件,用于设置参考文献处理时的选项。
一个bibmap程序,用于在后端处理参考文献数据。

bibmap宏包加载了natbib,chapterbib宏包,用于latex参考文献生成,bibmap程序类似bibtex程序用于处理参考文献数据。

bibmap宏包的工作原理有点类似biblatex,但又是极度简化的,目的是直接利用现有的latex宏包(比如natbib,chapterbib),避免像biblatex那样重写一整套的解决方案。而后端程序的运作又有点类似bibtex,但bibmap格式化文献表需要的样式文件时python数据和代码,因此是非常简单的,目的是让用户可以方便的设置参考文献格式,而不用去设计语法复杂的bst文件。

目前附带的bibmap程序是用python代码写的,后期会打包为可执行程序。

\subsection{bibmap的两大核心功能}

\subsubsection{参考文献生成}

参考文献生成类似于传统基于bibtex的方法。主要分如下几步:

1. 写tex文件源代码,其中参考文献相关宏包使用bibmap,无需再加载natbib。也不能使用biblatex,因为这是传统的方法,使用biblatex就冲突了。
可以为bibmap宏包设置一些选项。文档中正常引用文献,在需要生成文献表的地方加入命令\verb|\bibliography{bibfile}|

2. 第一遍xelatex编译tex源代码比如:
\verb|xelatex test.tex|,
第二遍用bibmap程序运行,命令为:
\verb|python biblatex-map.py test|,
第三和四遍使用xelatex编译tex源代码
\verb|xelatex test.tex|。
则能生成满足格式要求的参考文献表。

bibmap宏包利用了natbib等宏包来形成合适的参考文献引用标注标签,利用bibmap程序生成格式化的thebibliography环境。
一定程度上,bibmap程序是bibtex程序的替代,且它的参考文献格式设置要比bst文件简单得多。而bibmap宏包则只是对natbib等宏包的集成和利用,通过一些方便的选项设置接口,简化一些格式选择。

\subsubsection{bib文件修改}

bibmap程序的另一大功能是对bib文件的数据进行修改。这可以与tex文档相关,也可以不相关。不相关时可以直接利用bibmap程序对指定bib文件做指定格式的数据处理,相关时则利用指定的格式对bib文件做类似biblatex和biber做的动态数据修改。

bibmap程序对bib文件的数据修改,直接借鉴biblatex的设计,可以说是一套python的重新实现,使用逻辑基本一致。
可以对bib文件的条目和域做非常细致的处理和修改。

\section{参考文献生成的详细说明}

\subsection{bibmap宏包选项}


\subsection{参考文献生成示例}


\subsection{参考文献格式化设置说明}

介绍怎么写文献格式化的py文件


\section{bib文件修改的详细说明}

\subsection{bib文件修改示例}


\subsection{bib文件修改设置说明}

介绍怎么写数据修改的py文件


\section{bibmap程序基本用法}

\subsection{bibmap程序输入参数}

biblatex-map.py

filename 单个输入文件的文件名,可带后缀名如bib或aux,无后缀名时默认为辅助文件.aux

[-h] 输出帮助

[-a AUXFILE] 辅助文件的文件名,可带后缀名.aux,如果filename已经设置aux文件则无效

[-b BIBFILE] 文献数据库文件名,可带后缀名.bib,如果filename已经设置bib文件则无效

[-s STYFILE] 设置文献样式文件的文件名,可带后缀名.py,不给出则使用默认样式文件

[-m MAPFILE] 数据库修改设置文件文件名,可带后缀名.py,不给出则使用默认设置文件

[--nofmt] 给出该选项则不做格式化输出

[--nobdm] 给出该选项则不做不做bib数据修改

其中涉及到三种文件:

一是aux文件,如果是要得到格式化的文献表,那么这是最重要的文件,由tex编译生成,当使用bibmap宏包时,可以通过宏包选项设置样式文件,而bib文件通过bibliography命令也会在该文件中指出。

二是bib文件,这是参考文献数据源文件,可以由通过bibliography命令在aux文件内给出,也可以直接利用选项给出。

三是py文件,这是用于设置数据修改和文献格式化的文件,是python代码,通常bibmap*.py是用于bib文件数据修改的。
而bibstyle*.py是用于格式化文献表的。



\subsection{bib文件数据修改}

直接在命令行输入脚本及其参数:

\begin{example}{bib文件数据修改命令-默认情况}{code:bib:modify}
\begin{pycode}
python biblatex-map.py biblatex-map-test.bib
\end{pycode}
\end{example}

此时,bibmap读取biblatex-map-test.bib文件,并根据默认的数据修改设置bibmapdefault.py做修改,此时还会自动的做格式化后的文献表输出。

\begin{example}{bib文件数据修改命令-不输出格式化文献表}{code:bib:modifya}
\begin{pycode}
python biblatex-map.py biblatex-map-test.bib --nofmt
\end{pycode}
\end{example}

此时不再输出格式化后的文献表。

\begin{example}{bib文件数据修改命令-指定数据修改设置}{code:bib:modifyb}
\begin{pycode}
python biblatex-map.py biblatex-map-test.bib --nofmt -m bibmapaddkw.py
\end{pycode}
\end{example}

此时使用指定的数据修改设置bibmapaddkw.py代替默认的bibmapdefault.py对数据库bib文件做修改。



\subsection{参考文献格式化输出}

直接在命令行输入脚本及其参数:

\begin{example}{参考文献格式化命令-默认情况}{code:bib:fmt}
\begin{pycode}
python biblatex-map.py egtest
\end{pycode}
\end{example}

此时输入一个辅助文件egtest.aux,其它所有的参数根据对egtest.aux的解析来获取,如果没有解析到,若存在默认的设置,则使用默认的设置文件。若没有默认设置,则可以通过可选参数来指定:

\begin{example}{参考文献格式化命令-指定bib文件}{code:bib:fmta}
\begin{pycode}
python biblatex-map.py egtest -b biblatex-map-test.bib
\end{pycode}
\end{example}

当aux文件未给出格式化设置文件时,也可以用-s选项给出,格式化设置文件(即文献样式文件),比如
\begin{example}{参考文献格式化命令-指定样式文件}{code:bib:fmta}
\begin{pycode}
python biblatex-map.py egtest -s bibstyleauthoryear.py
\end{pycode}
\end{example}






\end{document}
